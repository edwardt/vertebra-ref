% Copyright 2008, Engine Yard, Inc.
\section{Overview}

Before we get too deeply into the concepts behind Vertebra, we should probably explore the concepts behind the core technologies that we leverage.  Our primary dependency is XMPP\index[tech]{XMPP|see{XML Messaging and Presence Protocol}}\index[tech]{XML Messaging and Presence Protocol}, as defined by the Internet Engineering Task Force\index[subject]{Internet Engineering Task Force} (IETF)\index[subject]{IETF|see{Internet Engineering Task Force}}.  It is fully documented in RFCs 3920\footnote{IETF RFC 3920: XMPP Core, \url{http://www.xmpp.org/rfcs/rfc3920.html}} and RFC 3921\footnote{IETF RFC 3921: XMPP IM, \url{http://www.xmpp.org/rfcs/rfc3921.html}}.

XMPP brings a number of advantages that are particularly helpful for implementing a distributed application like Vertebra.  In no particular order, some of them are:

\begin{itemize}
        \item Standardized Encoding (XML\index[tech]{Extensible Markup Language}\index[tech]{XML|see{Extensible Markup Language}})\footnote{W3C XML, \url{http://www.w3.org/TR/REC-xml/}}
        \item Transport Level Security (via TLS\index[tech]{Transport Layer Security}\index[tech]{TLS|see{Transport Layer Security}})\footnote{IETF RFC 4346: The Transport Layer Security (TLS) Protocol, \url{http://tools.ietf.org/html/rfc4346}}
        \item Authentication
        \item Automatic Federation (S2S + DNS)
        \item Federated Authentication (S2S Dialback)
        \item Client-side Redundancy (DNS SRV Records)
        \item International Text Support (UTF-8)
        \item Built-In Extensibility (Namespaces in XML\footnote{Namespaces in XML, \url{http://www.w3.org/TR/2006/REC-xml-names-20060816/}})
        \item Ordered Delivery
        \item Distributed Server Architecture (S2S)
        \item Sensible Message Semantics for Agents (IQ Delivery)
        \item Mechanism for Creating Server Extensible Protocols (IQ Server Handlers)
\end{itemize}
